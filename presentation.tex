\documentclass{beamer}

%\usepackage{tikz-cd}
\usepackage{quiver}
\usepackage{amsthm}
%\setbeamertemplate{theorems}[numbered]



%Information to be included in the title page:
\title{Localizations of Models of Dependent Type Theory}
\author{Matteo Durante \\ Advisor: Hoang-Kim Nguyen \\ Coadvisor: Denis-Charles
Cisinski}
\institute{Regensburg University}

\DeclareMathOperator{\Id}{\mathsf{Id}}
\DeclareMathOperator{\sfC}{\mathsf{C}}
\DeclareMathOperator{\N}{\mathbb{N}}
\DeclareMathOperator{\cC}{\mathcal{C}}
\DeclareMathOperator{\Ob}{\mathsf{Ob}}
\DeclareMathOperator{\ft}{ft}
\DeclareMathOperator{\app}{\mathsf{app}}


\begin{document}

\newtheorem{thm}{Theorem}[section]
\newtheorem{defn}{Definition}[section]
\newtheorem{conj}{Conjecture}[section]
\newtheorem{lem}{Lemma}[section]

\frame{\titlepage}

\begin{frame}
  \frametitle{Introduction}

  Dependent Type Theory is a candidate foundation for mathematics. Why is it
  interesting?

  \begin{enumerate}
    \item closely linked to \emph{computations} and \emph{computer science},
      thereby allowing the creation of programming languages and software tools
      like Coq, Lean and Agda to formalize and check mathematical reasoning and
      even automatically generate some tedious bits;
    \item sufficient by itself, unlike Set Theory and Propositional Calculus;
    \item \emph{proofs} are internal objects;
    \item allows a more nuanced concept of \emph{equality}, where a \emph{proof of
      equality} expresses in what sense two objects are equal.
  \end{enumerate}
\end{frame}

\begin{frame}
  \frametitle{Dependent Type Theory}
  Dependent Type Theory talks about \emph{dependent types $A$ over contexts
  $\Gamma$} and their \emph{terms $x:A$}.

  It gives \emph{structural rules} to specify how to substitute variables and
  \emph{logical rules} to construct new types and their terms
  from old and how to carry out computations
  through extra logical structure, like \emph{$\Sigma$-types}
  $\Sigma(A,B)$ (corresponding to coproducts), \emph{$\Pi$-types} $\Pi(A,B)$
  (also called \emph{function types}) and \emph{$\Id$-types} $\Id_A$.

  Problem: providing a model of Dependent Type Theory is hard because we have to
  deal with a lot of bureaucracy, especially from the structural rules.

  Solution: defining an algebraic model so that we can build new models from
  them.
\end{frame}

\begin{frame}
  \frametitle{Contextual Categories}

  \begin{definition}
    A category $\sfC$ with:
    \begin{enumerate}
      \item a grading on objects $\Ob\sfC=\coprod_{n\in\N}\Ob_n\sfC$, where
        $\Ob_0\sfC$ has only one element, which is the terminal;
      \item a map $\ft_n\colon\Ob_{n+1}\sfC\rightarrow\Ob_n\sfC$ for each
        $n\in\N$;
      \item for each object $\Gamma.A\in\Ob_{n+1}\sfC$ a dependent projection
        $p_A\colon\Gamma.A\rightarrow\ft(\Gamma.A)=\Gamma$;
      \item for each morphism $f\colon\Delta\rightarrow\Gamma$ and
        $p_A\colon\Gamma.A\rightarrow\Gamma$, a functorial choice of a pullback
        square;
        \[\begin{tikzcd}[ampersand replacement=\&]
          {\Delta.f^*A} \& {\Gamma.A} \\
          \Delta \& \Gamma
          \arrow["{p_A}", from=1-2, to=2-2]
          \arrow["{p_{f^*A}}"', from=1-1, to=2-1]
          \arrow["{q(f,A)}", from=1-1, to=1-2]
          \arrow["f"', from=2-1, to=2-2]
        \end{tikzcd}\]
    \end{enumerate}
  \end{definition}
  
  This only models the structural rules.
\end{frame}

\begin{frame}
  \frametitle{Extra Structure}
  Modeling logical rules requires some extra structure. $\Id$-types require:
  \begin{enumerate}
    \item from $\Gamma.A$ an object $\Gamma.A.A.\Id_A$;
    \item \ldots
  \end{enumerate}
  $\Pi$-types require:
  \begin{enumerate}
    \item from $\Gamma.A.B$ an object $\Gamma.\Pi(A,B)$;
    \item a map $\app_{A,B}\colon\Gamma.\Pi(A,B).A\rightarrow\Gamma.A.B$
      modeling function application;
    \item \ldots
  \end{enumerate}
  The data needed to model the logical rules we mentioned are
  called $\Sigma$-structure, $\Id$-structure and $\Pi$-structure.

  \begin{defn}
    A \emph{categorical model of type theory} is a contextual category $\sfC$
    with $\Sigma$, $\Id$ and $\Pi$ structure. The associated category with
    structure-preserving functors as maps is $Cxl_{\Sigma,\Id,\Pi}$.
  \end{defn}
\end{frame}

\begin{frame}
  \frametitle{Internal Languages Conjecture}

  Dependent type theories also have models provided by homotopy theory, like
  the category of $\infty$-groupoids. In
  particular, it has been conjectured that Dependent Type Theory is the
  \emph{internal language of $\infty$-categories}.

  \begin{conj}
    The horizontal maps by quasi-localizing contextual categories induce
    equivalences of $\infty$-categories.
    \[\begin{tikzcd}[ampersand replacement=\&]
      {Cxl_{\Sigma,1,\Id,\Pi}} \& {LCCC_\infty} \\
      {Cxl_{\Sigma,1,\Id}} \& {Lex_{\infty}}
      \arrow[from=1-1, to=2-1]
      \arrow[from=1-2, to=2-2]
      \arrow[from=2-1, to=2-2]
      \arrow[from=1-1, to=1-2]
    \end{tikzcd}\]
  \end{conj}

  A proof by Uemura and Nguyen has recently been published on arxiv and one
  hopes to build from this an equivalence between $Cxl_{HoTT}$ and
  $ElTopos_\infty$.
\end{frame}

\end{document}

