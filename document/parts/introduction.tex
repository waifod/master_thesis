\chapter*{Introduction}
\addcontentsline{toc}{chapter}{Introduction}

The theme of the foundations of mathematics is one which since its inception in
the nineteenth century has seen a rich history, with a crisis and a change in
perspective due to G{\"o}del's theorems. Various theories have since then been
devised to serve this role and the most notorious and successful among them is
Propositional Calculus with
Set Theory \cite{Jec13}, in the form of ZFC. This is however an
unsatisfactory foundation
for various reasons, chiefly among them the fact that every object is a set and
its treatment of equality, which lead to some rather counterintuitive
conclusions.

An alternative foundation which has become of interest in the past decades is
Type Theory, whose origins we can trace back to Russel \cite{Rus08,RW97},
who saw it as a tool to
block some paradoxes of Set Theory,
namely the existence of the set of all sets. This theory allows us to talk about
the \emph{type} of a mathematical
object and make equality a local concept (that is making sense only in terms of
two elements of the same type) and a more structured one, meaning that we can
talk about in what sense two objects are equal. Furthermore, proofs are
internalized by the theory as elements of types.

An important variant is Martin-L{\"o}f Type Theory \cite{ML84}, which is
intuitionistic and introduces \emph{dependent types}, that is types which are
allowed to depend on elements of other types. This is fairly convenient in
mathematical practice, where often the type of an object depends on other
objects, like in the statement ``for any field
$k$, let $V$ be a $k$-vector space''. This version was meant
to capture a notion of computation, exploiting the link with Computer Science
made evident by \emph{System F} \cite{Gir89}, a formalism for parametric
polymorphism in programming languages. This connection
has made it extremely interesting, as it led to the
creation of proof assistants like Coq, Lean, Agda and others. These are software
tools through which we can formalize and check mathematical reasoning, but also
automatically produce proofs of statements, thereby lifting a heavy burden from
the shoulders of mathematicians who often skip the ``obvious'' but tedious
details of proofs. In particular, Coq rose to notoriety because it was used to
prove the \emph{Four-Color Theorem} \cite{Gon08} and recently a project
was born to build the Lean Mathematical Library \cite{mat20}.

While it was clear early on that there was a link between categories and type
theories, leading to various notions of categorical models (like
Cartmell's \emph{contextual categories} \cite{Car78}) allowing a supposedly
equivalent approach to the study of these foundations, a central piece of this
connection still has not been provided, namely the \emph{Initiality Conjecture}
\cite{Str91}.
This would allow us to switch between algebraic and syntactic statements freely,
however even providing a rigorous statement has proven difficult as it would
require a precise definition of what a type theory is, which constitutes an
object of current research \cite{BHL20}.

Relatedly, further interest came from the Homotopy Theory community when
Hofmann-Streicher \cite{HS98} showed that the category of groupoids $\Gpd$ could
provide a model for an extension of Martin-L{\"o}f Type Theory, which only
became stronger because of the development of Homotopy Type Theory and its
simplicial model in the category of $\infty$-groupoids $\Gpdi$ by
Voevodsky \cite{KL12}, meaning that we may prove statements about Homotopy
Theory by working synthetically through Type Theory, leading to new proofs of
older theorems like the Black-Massey Theorem \cite{HFLL16}. Since then there has
been an effort to unify the
fields of $\infty$-Category Theory and Type Theory and thereby provide a
satisfactory foundation for mathematics with homotopical semantics. Such link is
obvious to the researchers in the field, but so far we have struggled to
progress: indeed, to unify the two we would need a procedure to construct a type
theory from an $\infty$-category and viceversa, however the first direction is
fairly problematic and a construction was provided only under the hypothesis of
local Cartesian closure and local presentability \cite{Shu14}.

This thesis focuses on the other direction, from a type theory to an
$\infty$-category, which itself is known only partially. Specifically, we aim to
prove the following result by Kapulkin \cite[Thm.\ 9.3.17]{Kap14} using
Cisinski's theory on localizations of $\infty$-categories \cite{Cis19} and
thereby avoiding the less general theory of frames \cite{Szu14,KS15}.

\begin{finalthm}
  Let $\bfT$ be a dependent type theory with $\Sigmas$-, $\Ids$- and
  $\Piext$-types. The localization of the nerve of $\Syn{T}$ at bi-invertible
  maps $L(N(\Syn{T}))$ is a locally Cartesian closed $\infty$-category.
\end{finalthm}

Following this result, Kapulkin and Lumsdaine formulated the \emph{Internal
Languages Conjecture} \cite[Conj.\ 3.7]{KL16}, implying an equivalence between
the $\infty$-categories of properly structured $\infty$-categories and the ones
of contextual categories with some extra data. This conjecture
was recently proven by Nguyen-Uemura \cite{NU22}. Current research aims to
extend this result by proving an equivalence between an $\infty$-category of
contextual categories modeling Homotopy Type Theory and the one of elementary
$\infty$-toposes, whose definition is still under discussion, however we already
know that $\infty$-toposes do provide models for it \cite{Shu19}.

\begin{organization}
  We begin in the first chapter by presenting our setting. Namely, in
  Chapter \ref{chapter1} we specify the foundations we use, the structural rules of our
  dependent type theories and the logical rules we will be considering; after
  this, we present the background material pertaining to the categorical side,
  introducing the syntactic category $\Syn{T}$, the Initiality Conjecture
  \ref{initconj} and various constructions
  with their type-theoretical meaning. In Chapter \ref{chapter2} we follow
  a later paper by Kapulkin \cite{Kap17} to produce
  extensions of the categorical data modeling the logical rules which we will
  and in Chapter
  \ref{chapter3} we introduce Cisinski's theory of localizations of
  $\infty$-category, leading to Theorem \ref{7616}, which is instrumental in our
  objective, and generalizing the theory of model categories. Finally, in
  Chapter \ref{chapter4} we proceed to prove our result by introducing a
  fibrational structure on $\Syn{T}$ \cite{AKL15} and some preliminary lemmas
  which allow us
  to apply the aforementioned theorem to $\Syn{T}$ and prove our result,
  whose statement we provide only at the end. We conclude with a
  recap of what we have done and presenting some progress which followed
  Kapulkin's proof.
\end{organization}
