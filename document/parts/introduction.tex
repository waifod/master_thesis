\chapter*{Introduction}
\addcontentsline{toc}{chapter}{Introduction}

The foundations of mathematics have seen a rich history since their inception in
the nineteenth century, including a crisis and a fundamental change in
perspective due to Gödel's theorems. Various theories have since been
devised to serve as foundations. The most notorious and successful among them is
Set Theory in the form of ZFC \cite{Jec13}. However, this is an
unsatisfactory foundation for various reasons. Chief among them are the fact that every object must be a set and
the treatment of equality, which lead to counterintuitive
conclusions.

An alternative foundation that has gained interest in recent decades is
Type Theory. Its origins trace back to Russell \cite{Rus08,RW97},
who developed it as a tool to
block certain paradoxes of Set Theory,
namely the existence of the set of all sets. Type Theory allows us to talk about
the \emph{type} of a mathematical object and treats equality as a local concept
(making sense only for two elements of the same type). Moreover, equality becomes
more structured, allowing us to discuss in what sense two objects are equal.
Furthermore, proofs are internalized by the theory as elements of types.

An important variant is Martin-L{\"o}f Type Theory \cite{ML84}, which is
intuitionistic and introduces \emph{dependent types}: types that are
allowed to depend on elements of other types. This is convenient in
mathematical practice, where the type of an object often depends on other
objects. For example, consider the statement ``for any field
$k$, let $V$ be a $k$-vector space''. This version was designed
to capture a notion of computation, exploiting the link with Computer Science
made evident by \emph{System F} \cite{Gir89}, a formalism for parametric
polymorphism in programming languages. This connection
has made Type Theory particularly interesting, leading to the
creation of proof assistants such as Coq, Lean, and Agda. These software
tools allow us to formalize and verify mathematical reasoning, and can also
automatically generate proofs of statements. They thereby lift a heavy burden from
mathematicians, who often skip the ``obvious'' but tedious
details of proofs. Notably, Coq rose to prominence when it was used to
prove the \emph{Four-Color Theorem} \cite{Gon08}. More recently, a project
was launched to build the Lean Mathematical Library \cite{mat20}.

The link between categories and type theories was recognized early on. This led to various notions of categorical models, such as
Cartmell's \emph{contextual categories} \cite{Car78}, which allow a supposedly
equivalent approach to studying these foundations. However, a central piece of this
connection remains unproven: the \emph{Initiality Conjecture}
\cite{Str91}.
This conjecture would allow us to switch freely between algebraic and syntactic statements.
Yet even providing a rigorous statement has proven difficult, as it requires
a precise definition of what constitutes a type theory, a question that remains an active area of research \cite{BHL20}.

Further interest came from the Homotopy Theory community when
Hofmann-Streicher \cite{HS98} showed that the category of groupoids $\Gpd$ could
provide a model for an extension of Martin-L{\"o}f Type Theory. This connection
became even stronger with the development of Homotopy Type Theory and its
simplicial model in the category of $\infty$-groupoids $\Gpdi$ by
Voevodsky \cite{KL12}. This means we can prove statements about Homotopy
Theory by working synthetically through Type Theory, leading to new proofs of
older theorems such as the Black-Massey Theorem \cite{HFLL16}. 

Since then, there has
been an effort to unify the
fields of $\infty$-Category Theory and Type Theory, thereby providing a
satisfactory foundation for mathematics with homotopical semantics. Such a link is
obvious to researchers in the field, but progress has been limited.
Indeed, unifying the two requires a procedure to construct a type
theory from an $\infty$-category and vice versa. However, the first direction is
problematic, and a construction was provided only under the restrictive hypotheses of
local Cartesian closure and local presentability \cite{Shu14}.

This thesis focuses on the other direction, from a type theory to an
$\infty$-category, which itself is known only partially. Specifically, we aim to
prove the following result by Kapulkin \cite[Thm.\ 9.3.17]{Kap14} using
Cisinski's theory on localizations of $\infty$-categories \cite{Cis19} and
thereby avoiding the less general Frame Theory \cite{Szu14,KS15}.

\begin{finalthm}
  Let $\bfT$ be a dependent type theory with $\Sigmas$-, $\Ids$- and
  $\Piext$-types. The localization of the nerve of $\Syn{T}$ at bi-invertible
  maps $L(N(\Syn{T}))$ is a locally Cartesian closed $\infty$-category.
\end{finalthm}

Following this result, Kapulkin and Lumsdaine formulated the \emph{Internal
Languages Conjecture} \cite[Conj.\ 3.7]{KL16}. This conjecture implies an equivalence between
the $\infty$-categories of properly structured $\infty$-categories and those
of contextual categories with appropriate extra data. This conjecture
was recently proven by Nguyen-Uemura \cite{NU22}. 

Current research aims to
extend this result further. Specifically, researchers seek to prove an equivalence between an $\infty$-category of
contextual categories modeling Homotopy Type Theory and the category of elementary
$\infty$-toposes. While the definition of elementary $\infty$-toposes is still under discussion,
we already know that $\infty$-toposes do provide models for Homotopy Type Theory \cite{Shu19}.

\begin{organization}
  We begin in Chapter \ref{chapter1} by presenting our setting. Specifically, 
  we specify the foundations we use, the structural rules of our dependent type 
  theories, and the logical rules we will be considering. We then present the 
  background material pertaining to the categorical side, introducing the 
  syntactic category $\Syn{T}$, the Initiality Conjecture \ref{initconj}, and 
  various constructions with their type-theoretical meaning.
  
  In Chapter \ref{chapter2}, we follow a later paper by Kapulkin \cite{Kap17} 
  to produce extensions of the categorical data modeling the logical rules we 
  consider.
  
  Chapter \ref{chapter3} introduces Cisinski's theory of localizations of 
  $\infty$-categories, leading to Theorem \ref{7616}, which is instrumental to 
  our objective and generalizes the theory of model categories.
  
  Finally, Chapter \ref{chapter4} proceeds to prove our main result by 
  introducing a fibrational structure on $\Syn{T}$ \cite{AKL15} and some 
  preliminary lemmas. These allow us to apply the aforementioned theorem to 
  $\Syn{T}$ and prove our result, whose statement we provide only at the end. 
  We conclude with a summary of our work and some developments that followed 
  Kapulkin's proof.
\end{organization}
